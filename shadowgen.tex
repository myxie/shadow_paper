\documentclass[a4paper,11pt]{article} 
\usepackage[utf8]{inputenc}
\usepackage{subfiles}
\usepackage[margin=1.5cm]{geometry}
\usepackage{cite}
\usepackage{caption}
\usepackage{listings}
\usepackage{multicol}
\usepackage{graphicx}
\graphicspath{{fig/}}
% \usepackage[hidelinks=true]{hyperref}
\usepackage{color,soul}
\usepackage{amsmath}
\usepackage{acronym}
\usepackage{subcaption}
\usepackage{ctable}
\usepackage{ragged2e} % provides \RaggedLeft\ usepackage{amssymb}
\usepackage{xcolor}
\usepackage{titlesec}
\usepackage{multicol}

% \usepackage{titlesec}

\usepackage{palatino}
\begin{document}

\section*{\textit{shadowgen}}
\label{sec:shadowgen}
\subsection*{Background}
\label{sec:background}

\textit{shadowgen} is a utility built into the \textit{shadow} framework to
generate workflows that are reproducible and interrogable. It is built to
generate a variety of workflows that have been documented and characterised in
the literature in a way that augments current techniques, rather than
replacing them entirely. 

This includes the following: 

\begin{itemize}
	\item Python code that runs the GGen graph
	generator\footnote{https://github.com/perarnau/ggen}, which produces graphs in a variety of shapes and sizes based on provided parameters. This was originally designed to produce task graphs to test the performance of DAG scheduling algorithms
	\item DAX Translator: This takes the commonly used Directed Acyclic XML (DAX)
	file format, used to generate graphs for Pegasus, and translates them into
	the \textit{shadow} format. Future work will also interface with the
	WorkflowGenerator code that is based on the work conducted
	in~\cite{bharathi2008}, which generates DAX graphs. 
	\item DALiuGE/EAGLE Translator: EAGLE logical graphs must be unrolled into
	Physical Graph Templates (PGT) before they are in a DAG that can be scheduled in
	\textit{shadow}. \textit{shadowgen} will run the DALiUGE unroll code, and then
	convert this PGT into a \textit{shadow}-based JSON workflow. 
\end{itemize}

Moving forward, shadow will also use the specifications in
hpconfig~\footnote{github.com/myxie/hpconfig}, which are class-based
descriptions of different hardware (e.g. \texttt{class XeonPhi} ) and
facilities (e.g \texttt{class PawseyGalaxy}) used in HPC. The idea behind hpconfig is
the classes can be used to quickly `unwrap' into a large cluster or system,
without having large JSON files in the repository or on disk; they also help
with readability, as the specification data is represented clearly as class
attributes. 

\subsection*{Existing approaches}
\label{ssec:existing}
A majority of work published in workflow scheduling will use workflows
generated using the approach laid out in~\cite{bharathi2008}. The five
workflows described in the paper (Montage, CyberShake, Epigenomics, SIPHT and
LIGO) had their task runtimes, memory and I/O rates profiled, from which they
created a WorkflowGenerator
tool\footnote{https://github.com/pegasus-isi/WorkflowGenerator}. This tool
uses the distribution sizes for runtime etc., without requiring any
information on the hardware on which the workflows are being `scheduled'. This
means that the characterisation is only accurate for that particular hardware,
if those values are to be used across the board; testing on heterogeneous
systems, for example, is not possible unless the values are to be changed.

This is dealt with in varied ways across the literature. For
example,~\cite{rodriguez2018} use the distributions
from~\cite{bharathi2008} paper, and change the units from seconds to MIPS,
rather than doing a conversion between the two. Others use the values taken
from distribution and workflow generator, without explaining how their
runtimes differ between resources~\cite{abrishami2013,malawski2015}; Malawski
et al, discuss how `20 different workflow instances were generated
using parameters and task runtmime distributions from real workflow
traces', but they do not provide the parameters~\cite{malawski2015}. Recent research still uses the workflows identified in~\cite{bharathi2008,juve2013}, but use only the structure of the workflows, replacing the tasks with other computationally intensive examples~\cite{wang2019}.

\subsection*{Proposed addition to \textit{shadowgen}} 
The method I propose is a normalised-cost approach, in which the values
calculated for the runtime, memory, and I/O for each tasks is determined based
on the normalised size as profiled in~\cite{juve2013} and~\cite{bharathi2008}.
This way, the costs per-workflow are indicative of the relative length and
complexity of each task, and are more likely to transpose across different
hardware configurations than using the varied approaches in the literature. 


\begin{table}[htb]
    \centering
    \caption{Experimental conditions for workflow profiling in~\cite{juve2013}}
    \resizebox{\columnwidth}{!}{%
    \begin{tabular}{@{\extracolsep{4pt}}cccccccccccc@{}}
        \toprule
        Workflow &  Site &  Nodes &  Cores &  Cpu (GHz) &  Memory(GB) &  File
        System &  ioprof &  pprof &  Normal \\
        \midrule
        Montage &  Amazon EC2 &  1 &  8 &  Xeon@2.33 &  7.5 &  ext3 &  1:11:20 &  0:57:36 &  0:55:28 \\
        CyberShake &  TACC Ranger &  36 &  16 &  Opteron@2.3 &  32 &  Lustre &  16:35:00 &  12:20:00 & N/A \\
        Broadband &  Amazon EC2 & 1 & 8 & Xeon@2.33 &  7.5 &  ext3 &  1:31:20 &  1:21:16 &  1:20:03 \\
        Epigenome &  Amazon EC2 & 1 & 8 & Xeon@2.33 &  7.5 &  ext3 &  1:11:11 &  1:08:01 &  1:07:40 \\
        LIGO &  Syracuse SUGAR & 80 & 4 & Xeon@2.50 &  7.5 &  NFS &  1:48:11 &  1:47:38 &  1:41:13 \\
        SIPHT &  UW Newbio &  13 & 8 & Xeon@3.16 &  7.5 &  ext3 &  1:33:24 &  1:19:37 &  1:10:53 \\
        \bottomrule
    \end{tabular}%
    }
    \label{tab:profile_environment}
\end{table}

\begin{table}[htb]
    \centering
    \caption{Example profile of Montage workflow, as presented in~\cite{juve2013}}
    \resizebox{\columnwidth}{!}{%
    \begin{tabular}{@{\extracolsep{4pt}}cccccccccccc@{}}
        \toprule
        Job & Count & \multicolumn{2}{c}{Runtime}&  \multicolumn{2}{c}{
        I/O Read} & \multicolumn{2}{c} {I/O Write} & \multicolumn{2}{c}{Peak Memory} &
        \multicolumn{2}{c}{CPU Util}\\ \cmidrule{3-4} \cmidrule{5-6} \cmidrule{7-8} \cmidrule{9-10} \cmidrule{11-12}
           & &  Mean (s) & Std. Dev. & Mean (MB) & Std. Dev. & Mean (MB)& Std. Dev. &
           Mean (MB)& Std. Dev. & Mean (\%) & Std. Dev \\ 
        \midrule
        mProjectPP & 2102 & 1.73 & 0.09 & 2.05 & 0.07 & 8.09 & 0.31 & 11.81 & 0.32 & 86.96 & 0.03 \\
        mDiffFit & 6172 & 0.66 & 0.56 & 16.56 & 0.53 & 0.64 & 0.46 & 5.76 & 0.67 & 28.39 & 0.16 \\
        mConcatFit & 1 & 143.26 & 0.00 & 1.95 & 0.00 & 1.22 & 0.00 & 8.13 & 0.00 & 53.17 & 0.00 \\
        mBgModel & 1 & 384.49 & 0.00 & 1.56 & 0.00 & 0.10 & 0.00 & 13.64 & 0.00 & 99.89 & 0.00 \\
        mBackground & 2102 & 1.72 & 0.65 & 8.36 & 0.34 & 8.09 & 0.31 & 16.19 & 0.32 & 8.46 & 0.10 \\
        mImgtbl & 17 & 2.78 & 1.37 & 1.55 & 0.38 & 0.12 & 0.03 & 8.06 & 0.34 & 3.48 & 0.03 \\
        mAdd & 17 & 282.37 & 137.93 & 1102.57 & 302.84 & 775.45 & 196.44 & 16.04 & 1.75 & 8.48 & 0.11 \\
        mShrink & 16 & 66.10 & 46.37 & 411.50 & 7.09 & 0.49 & 0.01 & 4.62 & 0.03 & 2.30 & 0.03 \\
        mJPEG & 1 & 0.64 & 0.00 & 25.33 & 0.00 & 0.39 & 0.00 & 3.96 & 0.00 & 77.14 & 0.00 \\
        \bottomrule
    \end{tabular}%
    }
    \label{tab:montage_profile}
\end{table}

The proposed distribution of values would be derived from a table of
normalised values using a varation on min-max feature scaling for each
mean/std. deviation column in Table~\ref{tab:montage_profile}. The formula used to calculated each tasks' normalised values is described in Equation~\ref{eq:normalise}; the results of applying this to Table~\ref{tab:montage_profile} is shown in Table~\ref{tab:norm_montage} 

\begin{equation}
\label{eq:normalise}
	X^\prime = \frac{(X \times n_{task})-X_{min}}{X_{max}-X_{min}}
\end{equation}

\begin{table}[htb]
    \centering
    \caption{Normalised values for columns derived from Profiled values Table~\ref{tab:montage_profile}}
    \resizebox{\columnwidth}{!}{%
    \begin{tabular}{@{\extracolsep{4pt}}cccccccccccc@{}}
        \toprule
        Job & \multicolumn{2}{c}{Runtime}&  \multicolumn{2}{c}{
        I/O Read} & \multicolumn{2}{c} {I/O Write} & \multicolumn{2}{c}{Peak Memory} &
        \multicolumn{2}{c}{CPU Util}\\ \cmidrule{2-3} \cmidrule{4-5} \cmidrule{6-7} \cmidrule{8-9} \cmidrule{10-11}
           &  Mean (s) & Std. Dev. & Mean (MB) & Std. Dev. & Mean (MB)& Std. Dev. &
           Mean (MB)& Std. Dev. & Mean (\%) & Std. Dev \\ 
        \midrule
        		mProjectPP & 9.47 & 0.49 & 11.22 & 0.38 & 44.30 & 1.70 & 64.66 & 1.75 & 476.20 & 0.16 \\
		mDiffFit & 10.61 & 9.00 & 266.27 & 8.52 & 10.29 & 7.40 & 92.61 & 10.77 & 456.48 & 2.57 \\
		mConcatFit & 0.37 & 0.00 & 0.00 & 0.00 & 0.00 & 0.00 & 0.01 & 0.00 & 0.13 & 0.00 \\
		mBgModel & 1.00 & 0.00 & 0.00 & 0.00 & 0.00 & 0.00 & 0.03 & 0.00 & 0.25 & 0.00 \\
		mBackground & 9.42 & 3.56 & 45.78 & 1.86 & 44.30 & 1.70 & 88.65 & 1.75 & 46.32 & 0.55 \\
		mImgtbl & 0.12 & 0.06 & 0.06 & 0.02 & 0.01 & 0.00 & 0.35 & 0.02 & 0.15 & 0.00 \\
		mAdd & 12.50 & 6.11 & 48.83 & 13.41 & 34.34 & 8.70 & 0.70 & 0.08 & 0.37 & 0.00 \\
		mShrink & 2.75 & 1.93 & 17.15 & 0.30 & 0.02 & 0.00 & 0.18 & 0.00 & 0.09 & 0.00 \\
		mJPEG & 0.00 & 0.00 & 0.06 & 0.00 & 0.00 & 0.00 & 0.00 & 0.00 & 0.19 & 0.00 \\
        \bottomrule
    \end{tabular}%
    }
    \label{tab:norm_montage}
\end{table}

This approach would allow algorithm designers and testers to prescribe what units they are interested in (e.g. seconds, MIPS, or FLOP seconds for runtime, MB or GB for Memory etc.) whilst still retaining the relative costs of that task within the workflow. In the example of Table~\ref{tab:norm_montage}, it is clear that mAdd and mBackground are still the longest running and I/O intensive tasks, and so the units are less of a concern.

% \subsection*{The environment testing model}
% % hpconfig - python classes that can be translated to \textit{shadow} format to present a real-world indication of the resources. This is to help with simulations that may work off shadow, like TOpSim. 

% \subsection*{Reasons for being pedantic about task costs}

% As outlined by Tover et al.~\cite{tovar2018}, measuring resource consumption
% of single resources does not mean that we can expect that to be the same for
% future jobs; and, if we try to counter this by providing an `upper bound' on
% our task costs, this would result in underutilisation of the system.
% \begin{itemize}
% 	\item We do not want to risk underutilisation of the system
% \end{itemize}
\clearpage
\bibliographystyle{ieeetr}
\bibliography{papers}
\end{document}